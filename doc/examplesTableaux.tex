\documentclass[11pt,spanish,a4paper]{article}

\usepackage{amsmath}
\usepackage{amstext}
\usepackage{amsfonts}
\usepackage{amssymb}
\usepackage[spanish]{babel}
\usepackage[latin1]{inputenc}
\usepackage[top=20mm,bottom=20mm,left=20mm,right=20mm,nohead,nofoot]{geometry}

\usepackage{tikz}
\usetikzlibrary{arrows,chains,matrix,positioning,scopes,intersections,calc,trees,positioning,arrows,chains,shapes.geometric,%
  decorations.pathreplacing,decorations.pathmorphing,shapes,%
  matrix,shapes.symbols}

\usetikzlibrary{decorations.pathmorphing} % noisy shapes
\usetikzlibrary{fit}	
\usetikzlibrary{arrows}
\usetikzlibrary{backgrounds}	
\usetikzlibrary{arrows,chains,matrix,positioning,scopes}

\pagestyle{empty}


\newcommand{\LPO}{\ensuremath{\mathsf{LPO}}}
\newcommand{\lpo}{\ensuremath{\mathcal{L}_{\mathsf{LPO}}}}
\newcommand{\alpo}{\ensuremath{\mathsf{A}_{\lpo}}}
\newcommand{\tlpo}{\ensuremath{\mathsf{T}_{\lpo}}}
\newcommand{\flpo}{\ensuremath{\mathsf{F}_{\lpo}}}
\newcommand{\mo}{\ensuremath{\mathcal{M}}}
\newcommand{\dominio}{\ensuremath{\mathcal{D}}}%\mathfrak{D}}}
\newcommand{\interp}{\ensuremath{\mathcal{I}}}%\Im}}
\newcommand{\modelo}{\ensuremath{\mo = \langle \dominio, \interp\rangle}}
\newcommand{\constantes}{\ensuremath{\mathcal{C}}}
\newcommand{\variables}{\ensuremath{\mathcal{V}}}
\newcommand{\functores}{\ensuremath{\mathcal{F}}}
\newcommand{\predicados}{\ensuremath{\mathcal{P}}}



\begin{document}

\title{Ejemplos de tableaux anotados}
\date{\mbox{}}
\maketitle{}
\thispagestyle{empty}

\section{L�gica proposicional}
\label{sec:logica-proposicional}

\noindent{}\textbf{Satisfacibilidad.}
Comenzamos con un tableau para comprobar si la f�rmula $(p\to \lnot
p\lor \lnot q)\land p$ es satisfacible: 

  \begin{center}
\begin{tikzpicture}[scale=1,baseline=(r1.base),level/.style={sibling
    distance=3cm,level distance=1cm}] 
  [-,thick]
  \node[label=left:{1.}] (r1) {{$(p\to \lnot p\lor \lnot q)\land p$}}  
  child {    
    node[label=left:{2.},label=right:{$\alpha$,1}] {$p\to \lnot p\lor \lnot q$}  
    child {    
      node[label=left:{3.},label=right:{$\alpha$,1}] {$p$}
      child[sibling distance=4cm] {    
        node[label=left:{4.},label=right:{$\beta$,2}] {$\lnot p$}
     child {    
        node[label=right:{3,4}] {$\bigotimes$}
      } 
      } 
      child {    
        node[label=left:{5.},label=right:{$\beta$,2}] {$\lnot p\lor
          \lnot q$}
        child[level distance=1.5cm] {    
          node[label=left:{6.},label=right:{$\beta$,5}] {$\lnot p$}
          child {    
            node[label=right:{3,6}] {$\bigotimes$}
          } 
        } 
        child[level distance=1.5cm] {    
          node[label=left:{7.},label=right:{$\beta$,5}] {$\lnot q$}
        } 
      } 
   } 
  };
\end{tikzpicture}
  \end{center}
Como el tableau es abierto, la f�rmula es satisfacible. A partir de
los literales de la rama abierta definimos una interpretaci�n $\sigma$
que satisface $(p\to \lnot p\lor \lnot q)\land p$: $\sigma(p)=1$,
$\sigma(q)=0$. 

\bigskip{}

\noindent{}\textbf{Consecuencia l�gica.}
Comprobamos $\{p\to q\land r,\ \lnot(q\land t)\}\models
t\to\lnot p$ haciendo el tableau de las premisas y la negaci�n de la
conclusi�n. 
  \begin{center}    
\begin{tikzpicture}[scale=0.6,baseline=(r1.base),level/.style={sibling
    distance=5cm,level distance=1.4cm}]
  [-,thick]
  \node[label=left:{1.},label=right:{{(Prem.)}}] (r1) {{$p\to q\land r$}}  
  child {
    node[label=left:{2.},label=right:{{(Prem.)}}] {{$\lnot (q\land t)$}}
    child {
      node[label=left:{3.},label=right:{{(Negac. Conc.)}}] {{$\lnot (t\to \lnot p)$}}
      child {
        node[label=right:{$\alpha$,3},label=left:{4.}] {$t$}
        child {
          node[label=left:{5.},label=right:{$\alpha$,3}] {$\lnot\lnot p$}
          child {
            node[label=left:{6.},label=right:{$\sigma$,5}] {$p$}
            child {
              node[label=left:{7.},label=right:{$\beta$,1}] {$\lnot
                p$}
              child{node[label=right:{6,7}] {$\bigotimes$}}
            }
            child {
              node[label=left:{8.},label=right:{$\beta$,1}] {$q\land
                r$}
              child{
                node[label=left:{9.},label=right:{$\alpha$,8}] {$q$}
                child{
                  node[label=left:{10.},label=right:{$\alpha$,8}]
                  {$r$}
                  child{
                    node[label=left:{11.},label=right:{$\beta$,2}]
                    {$\lnot q$}
                    child{
                      node[label=right:{9,11}]
                      {$\bigotimes$}              }
                  }
                  child{
                    node[label=left:{12.},label=right:{$\beta$,2}]
                    {$\lnot t$}
                    child{
                      node[label=right:{4,12}]
                      {$\bigotimes$}
                  } 
                  } 
                }
              }
            }
          }
        }
      }
    }
  };
\end{tikzpicture}
\end{center}
Obtenemos un {tableau cerrado}, por tanto la conclusi�n es
consecuencia l�gica de las premisas.

\section{L�gica de predicados}
\label{sec:logica-de-predicados}

\noindent{}\textbf{Validez.}
Comprobamos si la f�rmula $\exists x (\lnot Px \land Qx) \to \exists y
    (Qy\to \lnot Py\land Ry)$ es v�lida haciendo el tableau de la
    negaci�n. 
\begin{center}
\begin{tikzpicture}[scale=.8,baseline=(r1.base),level/.style={sibling
    distance=6cm,level distance=1cm}]
  [-,thick]
  \node[label=left:{1.}] (r1) {$\lnot(\exists x (\lnot Px \land Qx) \to
    \exists y (Qy\to \lnot Py\land Ry))$}  
  child {
    node[label=left:{2.},label=right:{$\alpha$,1}] {$\exists x (\lnot
      Px \land Qx)$}
    child {
      node[label=left:{3.},label=right:{$\alpha$,1}] {$\lnot\exists y
        (Qy\to \lnot Py\land Ry)$}
      child {
        node[label=left:{4.},label=right:{$\delta$,2}] {$\lnot Pa\land
          Qa$}
        child {
          node[label=left:{5.},label=right:{$\alpha$,4}] {$\lnot Pa$}
          child {
            node[label=left:{6.},label=right:{$\alpha$,4}] {$Qa$}
            child {
              node[label=left:{7.},label=right:{$\gamma$,3}]
              {$\lnot(Qa\to \lnot Pa\land Ra)$}
              child {
                node[label=left:{8.},label=right:{$\alpha$,7}]
                {$Qa$}
                child {
                  node[label=left:{9.},label=right:{$\alpha$,7}]
                  {$\lnot (\lnot Pa\land Ra)$}
                  child {
                    node[label=left:{10.},label=right:{$\beta$,9}]
                    {$\lnot \lnot Pa$}
                    child {
                      node[label=below:{5,10}]
                      {$\bigotimes$}}}
                  child {
                    node[label=left:{11.},label=right:{$\beta$,9}]
                  {$\lnot Ra$}}}}}}}}}};
\end{tikzpicture}
\end{center}
La rama de la derecha queda abierta, por lo que la f�rmula inicial
es satisfacible. Por tanto, $\exists x (\lnot Px \land Qx)
\to \exists y (Qy\to \lnot Py\land Ry)$ no es v�lida. Podemos
construir una estructura $\modelo$ que la hace falsa: 
$\dominio=\{1\}$, $\interp(a)=1$, $\interp(Q)=\{1\}$,
$\interp(P)=\interp(R)=\varnothing$.

\bigskip{}

\noindent{}\textbf{Satisfacibilidad.}
Nos preguntamos si el conjunto $\{\forall x (Px\to Qx),\
  \exists x Px,\ \lnot \forall x Qx\}$ es satisfacible. Hacemos su
  tableau.  
   \begin{center}
      \begin{tikzpicture}[scale=0.8,baseline=(r1.base),level/.style={sibling
    distance=6cm,level distance=1cm}]
  [-,thick]
  \node[label=left:{1.}] (r1) {$\forall x (Px\to Qx)$}  
  child {
    node[label=left:{2.}] {$\exists x Px$}
    child {
      node[label=left:{3.}] {$\lnot \forall x Qx$}
      child {
        node[label=right:{$\delta$,2},label=left:{4.}] {$Pa$}
        child {
          node[label=left:{5.},label=right:{$\delta$,3}] {$\lnot Q b$}
          child {
            node[label=left:{6.},label=right:{$\gamma$,1}] {$Pa\to Qa$}
            child {
              node[label=left:{7.},label=right:{$\gamma$,1}] {$Pb\to
                Qb$}
              child {
                node[label=left:{8.},label=right:{$\beta$,6}] {$\lnot Pa$}
                child {
                  node[label=below:{4,8}] {$\bigotimes$}}}
              child {
                node[label=left:{9.},label=right:{$\beta$,6}] {$Q a$}
                child {
                  node[label=left:{10.},label=right:{$\beta$,7}] {$\lnot Pb$}}
                child {
                node[label=left:{11.},label=right:{$\beta$,7}] {$Qb$}
                child {
                  node[label=below:{5,11}] {$\bigotimes$}}}              
}}}}}}};
\end{tikzpicture}
\end{center}
El tableau es abierto, por lo que el conjunto propuesto es
satisfacible. La rama abierta est� formada por los nodos 1--7 y 9--10. Su
literales son $\{Pa,\lnot Qb,Qa,\lnot Pb\}$. Definimos la
estructura $\modelo$ con $\dominio=\{1,2\}$, $\interp(a)=1$,
$\interp(b)=2$, $\interp(P)=\{1\}$ y $\interp(Q)=\{1\}$. Se puede
comprobar que $\modelo$ satisface el conjunto de f�rmulas propuesto.  

\bigskip{}

\noindent{}\textbf{Consecuencia l�gica.}
  Para comprobar $\forall x (Px \to \forall y Rxy),\ \forall x \exists y \lnot
    Rxy \models \lnot Pa$ debemos hacer el tableau de las dos premisas
    y la negaci�n de la conclusi�n.
    \begin{center}
           \begin{tikzpicture}[scale=0.8,m
             baseline=(r1.base),level/.style={sibling 
               distance=6cm,level distance=1cm}]
  [-,thick]
  \node[label=right:{(Premisa)},label=left:{1.}] (r1) {$\forall x (Px\to \forall y Rxy)$}  
  child {
    node[label=right:{(Premisa)},label=left:{2.}] {$\forall x\exists
      y\lnot R xy$}
    child {
      node[label=right:{(Negac. concl.)},label=left:{3.}] {$\lnot
        \lnot Pa$}
      child {
        node[label=right:{$\sigma$,3},label=left:{4.}] {$Pa$}
        child {
          node[label=left:{5.},label=right:{$\gamma$,1}] {$Pa\to
            \forall y Ray$}
          child {
            node[label=left:{6.},label=right:{$\beta$,5}] {$\lnot Pa$}
            child {
              node[label=below:{4,6}] {$\bigotimes$}}}
          child {
            node[label=left:{7.},label=right:{$\beta$,5}] {$\forall y
              Ray$}
            child {
            node[label=left:{8.},label=right:{$\gamma$,2}] {$\exists y
              \lnot R
              ay$}
            child {
            node[label=left:{9.},label=right:{$\delta$,8}] {$\lnot R
              ab$}
            child {
            node[label=left:{10.},label=right:{$\gamma$,7}] {$R ab$}
            child {
              node[label=below:{9,10}] {$\bigotimes$}      
}}}}}}}}};
\end{tikzpicture}
    \end{center}
Dado que el tableau es cerrado, se verifica la relaci�n de
consecuencia l�gica propuesta.


\end{document}

%%% Local Variables:
%%% mode: latex
%%% TeX-master: t
%%% End:
